% Тут используется класс, установленный на сервере Papeeria. На случай, если
% текст понадобится редактировать где-то в другом месте, рядом лежит файл matmex-diploma-custom.cls
% который в момент своего создания был идентичен классу, установленному на сервере.
% Для того, чтобы им воспользоваться, замените matmex-diploma на matmex-diploma-custom
% Если вы работаете исключительно в Papeeria то мы настоятельно рекомендуем пользоваться
% классом matmex-diploma, поскольку он будет автоматически обновляться по мере внесения корректив
%

% По умолчанию используется шрифт 14 размера. Если нужен 12-й шрифт, уберите опцию [14pt]
\documentclass[14pt]{matmex-diploma}
\usepackage{algorithm}
\usepackage[noend]{algpseudocode}
\usepackage{amsthm}
\usepackage{amssymb}
\usepackage{amsmath}
\usepackage{hyperref}% http://ctan.org/pkg/hyperref
\usepackage{cleveref}% http://ctan.org/pkg/cleveref

%\documentclass[14pt]{matmex-diploma-custom}

\theoremstyle{definition}
\floatname{algorithm}{Алгоритм}
\newtheorem{definition}{Определение}
\newtheorem{theorem}{Теорема}
\newtheorem{lemma}{Лемма}
\newtheorem{corollary}{Следствие}
\crefname{lemma}{Лемме}{Леммы}
\newcommand{\overtext}[2]{\ensuremath{\stackrel{\mathrm{#1}}{\mathrm{#2}}}}
%\newtheorem{proof}{Доказательсво}
\begin{document}
% Год, город, название университета и факультета предопределены,
% но можно и поменять.
% Если англоязычная титульная страница не нужна, то ее можно просто удалить.
\filltitle{ru}{
    chair              = {Кафедра системного программирования},
    title              = {Поиск кратчайшего пути с контекстно-свободными ограничениями в помеченном графе},
    % Здесь указывается тип работы. Возможные значения:
    %   coursework - Курсовая работа
    %   diploma - Диплом специалиста
    %   master - Диплом магистра
    %   bachelor - Диплом бакалавра
    type               = {coursework},
    position           = {студента},
    group              = 344,
    author             = {Антополь Валерий Алексеевич},
    supervisorPosition = {ст. преп.\,к. ф.-м. н.},
    supervisor         = {Григорьев С.\,В.},
    %reviewerPosition   = {ст. преп.},
    %reviewer           = {Привалов А.\,И.},
    %chairHeadPosition  = {д.\,ф.-м.\,н., профессор},
    %chairHead          = {Хунта К.\,Х.},
%   university         = {Санкт-Петербургский Государственный Университет},
%   faculty            = {Математико-механический факультет},
%   city               = {Санкт-Петербург},
%   year               = {2013}
}
%\filltitle{en}{
%    chair              = {The Meaning of Life \\ Uselessness of Everything},
%    title              = {Empty subset as closed set},
%    author             = {Edelweis Mashkin},
%    supervisorPosition = {professor},
%    supervisor         = {Amvrosy Vibegallo},
%    reviewerPosition   = {assistant},
%    reviewer           = {Alexander Privalov},
%    chairHeadPosition  = {professor},
%    chairHead          = {Christobal Junta},
%}
\maketitle
\tableofcontents
% У введения нет номера главы
\section*{Введение}
    Структуры данных на основе графов часто используются в различных областях, например, в биоинформатике, логистике, графовых базах данных, в языках программирования. Одна из таких структур --- это граф, каждому ребру которого сопоставляется символ из некоторого конечного алфавита $\Sigma$. Такой граф называется помеченным.
    
    Для того, чтобы получить информацию из графа, нужно выполнять запросы, задающие класс путей в графе. Пути рассматриваются в виде строки из меток на ребрах. Такие запросы удобно представлять в виде грамматики: путь удовлетворяет запросу, если он принадлежит языку, который порождает заданная грамматика. Часто результатом запроса должно являться отношение на вершинах графа: если (n, m) принадледжит отношению, то существует путь из n в m, который распознается грамматикой. Это называется задачей достижимости с контекстно-свободными ограниениями.
    
    В некоторых областях, например, в логистике, недостаточно найти какой-либо путь (или его существование) между парой вершин, нужно найти кратчайший (или его длину). Существуют различные алгоритмы поиска кратчайшего пути в графе с контекстно-свободными ограничениями. Но некоторые из них требуют отсутствия отрицательных весов~\cite{Barrett} или ацикличности графа~\cite{p2pdyck}.
    
    В 2017 году был предложен алгоритм решения задачи достижимости в произвольном помеченном графе с контекстно-свободными ограничениями~\cite{DBLP:journals/corr/AzimovG17}. Его особенность в том, что большая часть необходимих вычислений - это перемножение матриц, поэтому он легко параллелится на GPGPU~(General-Purpose computing on
Graphics Processing Units). 
    
    В данной работе описывается модификация этого алгоритма для решения задачи поиска кратчайшйх путей между каждой парой вершин с контекстно-свободными ограничениями. В отличие от других алгоритов, эта модификация имеет слабые ограничения: единственное условие на граф --- отсутствие циклов отрицательного веса.
    
    
    
%Помеченный граф - это граф, каждому ребру которого сопоставляется символ из некоторого алфавита $\Sigma$.
%\theoremstyle{definition}
%\begin{definition}
%    $(min, +)$-произведение двух числовых матриц $A$, $B$ - это %числовая матрица C, такая что $C_{i,j} = min_k\{A_{i,k} %+ B_{k,j}\}$
%\end{definition}
\section{Постановка задачи}
    Целью данной работы является модификация алгоритма решения задачи достижимости с контекстно-свободными ограничениями в помеченном графе~\cite{DBLP:journals/corr/AzimovG17} для решения задачи поиска кратчайшйх путей между каждой парой вершин с контекстно-свободными ограничениями. Для достижения данной цели были поставлены следующие задачи:
    \begin{itemize}
        \item создать модификацию алгоритма;
        \begin{itemize}
            \item доказать корректность;
            \item дать оценку времени работы;
        \end{itemize}
        \item провести экспериментальное исследование модифицированного алгоритма.
    \end{itemize}
\section{Обзор}
    \subsection{Fast-BJM}
        Алгоритм использует методы динамического программирования и очередь с приоритетами, за счет этого достигается время работы $O(|V|^3|N||P|)$, где $V$ --- множество ребер в графе, $N$ --- множество нетерминалов в грамматике, $P$ - множество правил вывода в грамматике. Не работает с ребрами отрицательного веса, также в графе не должно быть петель.
        Существует модификация этого алгоритма, позволяющая распараллеливание~\cite{D-Fast-BJM}. При распараллеливании на $|V|$ узлов каждый узел завершит работу за время $O(|V|^2|N||P|)$.
        Также существует модификация, имеющая ту же временную сложность, что и оригинальный алгоритм, но позволяющая работать с ребрами отрицательного веса~\cite{Johnson-Fast-BJM}. 
        
    \subsection{Fast Bounded-Difference Min-Plus Product}
        Этот алгоритм решает задачу, близкую к задаче поиска кратчайших путей с контекстно-свободными ограничениями.
        Отличие состоит в том, что вес есть не у ребер графа, а у правил вывода в грамматике~\cite{DBLP:journals/corr/BringmannGSW17}.
        Он состоит из построения транзитивного замыкания графа с помощью произведения матриц~\cite{VALIANT1975308}, в котором операции над элементами матрицы взяты из  $(min, +)$-произведения.
        За счет введения дополнительных ограничений на элементы перемножаемых матриц получается достичь $O(|V|^{2.8603})$ времени работы.

    \subsection{Context-Free Path Querying by Matrix Multiplication}
        В этой статье предложили алгоритм решения задачи достижимости с контекстно-свободными ограничениями с использованием транзитивного замыкания графа~\cite{VALIANT1975308}. Также в этой статье модифицировали алгоритм построения транзитивного замыкания, уменьшив объем вычислений. Именно этот алгоритм решения задачи достижимости и будет модифицироваться в данной работе.
        
\section{Модификация}
    \subsection{Исходный алгоритм}
        Пусть $(S, N,T,P)$ --- контекстно-свободная грамматика в нормальной форме Хомского, где $N$ --- множество нетерминалов, $T$ --- множество терминалов, $P$ --- множество правил вывода, $S$ --- стартовый нетерминал.\\
        Пусть $(V,E)$ --- помеченный ориентированный мультиграф, где $V$ --- множество вершин, $E \subset V \times T \times V$ --- множество рёбер.\\
        Алгоритм состоит в построении транзитивного замыкания матрицы смежности графа, определяемой специальным образом.
        Элементами такой матрицы являются подмножества $N$. Пусть $a$ --- матрица смежности графа. Тогда если $A\in a_{i,j}$, то существует $i\pi j$ такой, что $w(\pi)\in~L(G_A)$, где $i\pi j$ --- путь в графе из вершины $i$ в вершину $j$, $w(\pi)$ --- слово, образуемое метками на пути $\pi$, $L(G)$ --- язык порожденный грамматикой $G$, $G_A$ --- грамматика, получающаяся из $G$ заменой стартового нетерминала на $A$.
        Введем операции на $N_1$ и $N_2$ --- произвольных подмножествах $N$:
        \begin{equation*}
            N_1 \times N_2 = \{A~|~\exists B \in N_1,~\exists C \in N_2 : A \rightarrow BC \in P\}
        \end{equation*}
        \begin{equation*}
            N_1 + N_2 = N_1 \cup N_2
        \end{equation*}
        Транзитивное замыкание матрицы $a$ определяется как $a^{cf} = \sum\limits_{i=1}^\infty a^{(i)}$, 
        где 
        \begin{equation*}
            a^{(i)} = 
            \begin{cases} 
                a,                                     & i=1\\
                a^{(i-1)} + a^{(i-1)}\times a^{(i-1)}, & i>1
            \end{cases}
        \end{equation*}

        
        Для транзитивного замыкания матрицы $a$ доказано~\cite{DBLP:journals/corr/AzimovG17} следующее:\\
        $A\in~a^{cf}_{i,j}$ тогда и только тогда, когда существует $i\pi j$ такой, что $w(\pi)\in~L(G_A)$.
        
        \begin{algorithm}[H]
        \begin{algorithmic}[1]
        \caption{Решение задачи достижимости}
        \label{alg:graphParse}
        \Function{contextFreePathQuerying}{D - граф, G - грамматика}
            \State{$n \gets$ количество вершин в $D$}
            \State{$E \gets$ множество ребер в $D$}
            \State{$P \gets$ множество правил вывода в $G$}
            \State{$T \gets$ матрица размера $n \times n$, в которой каждый элемент равен $\varnothing$}
            \ForAll{$(i,x,j) \in E$}
            \Comment{Инициализация матрицы}
                \State{$T_{i,j} \gets T_{i,j} \cup \{A~|~(A \rightarrow x) \in P \}$}
            \EndFor    
            \While{матрица $T$ меняется}
                \State{$T \gets T \cup (T \times T)$}
                \Comment{Вычисление транзитивного замыкания $T$} 
            \EndWhile
        \State \Return $T$
        \EndFunction
        \end{algorithmic}
        \end{algorithm}
    \subsection{Модифицированный алгоритм}
        Множество ребер графа теперь является подмножеством $V \times T \times V \times \mathbb{R}$ --- у ребер появляется вес. Элементами используемой матрицы являются функции такого типа: $a_{i,j}: N\rightarrow\mathbb{R}\cup\{\infty\}$.\\
        Элементами используемой матрицы являются подмножества: $N\times(\mathbb{R}\cup\{\infty\})$.\\
        На элементах матрицы водятся следующие операции:
        \begin{equation*}
            (N_1 \times N_2)(A) = \min\{N_1(B) + N_2(C)~|~A \rightarrow BC \in P\}
        \end{equation*}
        \begin{equation*}
            (N_1 + N_2)(A) = \min\{N_1(A), N_2(A)\}
        \end{equation*}
        Это в точности $(min, +)$-произведение.
        
        $a^i$ и $a^{cf}$ определяются как в исходном алгоритме, но с новыми операциями $(+)$ и $(\times)$.
        
        \begin{algorithm}[H]
        \begin{algorithmic}[1]
        \caption{Решение задачи поиска кратчайших путей}
        \label{alg:graphParseShortest}
        \Function{contextFreePathQuerying}{D - граф, G - грамматика}
            \State{$n \gets$ количество вершин в $D$}
            \State{$E \gets$ множество ребер в $D$}
            \State{$P \gets$ множество правил вывода в $G$}
            \State{$T \gets$ матрица размера $n \times n$, $T_{i,j}(A) = \infty$}
            \ForAll{$(i,x,j,w) \in E$}
            \Comment{Инициализация матрицы}
                \ForAll{$A~|~(A\rightarrow x) \in P$}
                    \State{$T_{i,j}(A) \gets \min{(T_{i,j}(A),w})$}
                \EndFor
            \EndFor    
            \While{матрица $T$ меняется}
                \State{$T \gets T + (T \times T)$}
                \Comment{Вычисление транзитивного замыкания $T$ и поиск кратчайших путей} 
            \EndWhile
        \State \Return $T$
        \EndFunction
        \end{algorithmic}
        \end{algorithm}
    \subsection{Доказательство корректности}   
       \subsubsection{Используемые обозначения}
           Запись $i\pi j$ означает, что $\pi$ --- это путь в графе от вершины $i$ к вершине $j$.\\
           $l(\pi)$ --- слово, образуемое конкатенацией меток на ребрах, входящих в путь $\pi$ в порядке прохождения.\\
           $w(\pi)$ --- длина пути $\pi$, то есть сумма весов входящих в него ребер.
           Путь $\pi$ называется подходящим для пары (A, k), если cуществует дерево вывода высоты $h\le k$ для строки $l(\pi)$ в $G_A=(N,\Sigma, P, A)$. Заметим, что если $\pi$ подходящий для пары $(A, k)$ то он также подходящий для пары $(A, n)$, где $n > k$.
       \begin{lemma}
       \label{le_any}
           Пусть $D=(V,E)$ --- граф, $G=(N,\Sigma, P, S)$ --- контекстно-свободная грамматика в нормальной форме Хомского. Тогда, для любых $i$, $j$, $k$ и нетерминала $A\in N$
       верно следующее: $a^k_{i,j}(A) \le w(\pi)$ для любого пути $i\pi j$ подходящего для пары $(A, k)$.
       \end{lemma}
       \begin{proof}
           По индукции.\\
           \textbf{База:} по построению $a^1$ = $a$. Все деревья вывода высоты 1 соотвествуют каждое своему ребру.\\
           \textbf{Переход:} Пусть p > 1. Предположим, что утверждение леммы выполняется для $k < p$. Рассмотрим $a^p_{i,j}(A)$.По определениям операций $(\times)$, $(+)$ и умножения матриц:
           \begin{equation}\label{apply_plus}
               a^p_{i,j}(A) = min(a^{p-1}_{i,j}(A), (a^{p-1}\times a^{p-1})_{i,j}(A))
           \end{equation}
           \begin{equation}\label{apply_times}
                (a^{p-1}\times a^{p-1})_{ij}(A) = \min_{r,A,B}\{a^{p-1}_{ir}(B) + a^{p-1}_{rj}(C)~|~\exists A\rightarrow BC \in P\}
           \end{equation}                                 
           По индукционному предположению $a^{p-1}_{i,j}(A) \le w(\pi)$ для всех подходящих паре $(A, p-1)$ путей $i\pi j$. Из этого и \eqref{apply_plus} следует, что $a^p_{i,j}(A) \le w(\pi)$ для всех $i\pi j$, подходящих $(A, p-1)$.
           
           Далее необходимо показать, что неравенство верно также и для путей, подходящих $(A, p)$, но не подходящих $(A, p-1)$. Рассмотрим такой путь $i\pi j$ и дерево вывода $l(\pi)$ в $G_A$ высоты $h=p$. Так как грамматика в нормальной форме Хомского, у корня есть ровно 2 ребенка. Они являются корнями подеревьев, являющихся по определению деревьями вывода некоторых строк $l_{left}$ в $G_B$ и $l_{right}$ в $G_C$ соответственно. Так как конкатенация $l_{left}$ и $l_{right}$ это $l(\pi)$, $i\pi j$ можно разбить на $i\pi_{left} r$ и $r\pi_{right} j$ так, чтобы выполнялось $l(\pi_{left}) = l_{left}$ и $l(\pi_{right}) = l_{right}$.
            Из индукционного предположения:
            \begin{equation*}
                a^{p-1}_{ir}(B) \le w(\pi_{left})                
            \end{equation*}
            \begin{equation*}
                a^{p-1}_{rj}(C) \le w(\pi_{right})                
            \end{equation*}
            Из этого и \eqref{apply_times}:
            \begin{equation*}
                (a^{p-1}\times a^{p-1})_{ij}(A) \le a^{p-1}_{ir}(B) + a^{p-1}_{rj}(C) \le w(\pi_{left}) + w(\pi_{right}) = w(\pi)
            \end{equation*}
            Таким образом:
            \begin{align*}
                a^p_{i,j}(A) &\le a^{p-1}_{ij}(A)                 &\le w(\pi) & \text{для}~i\pi j~\text{подходящих}~(A, p-1)\\
                a^p_{i,j}(A) &\le (a^{p-1}\times a^{p-1})_{ij}(A) &\le w(\pi) & \text{для}~i\pi j~\text{подходящих}~(A, p)~\text{, но не}~(A, p-1)
            \end{align*}
            Cледовательно:
            \begin{equation*}
                a^p_{i,j}(A) \le w(\pi)~\text{для}~i\pi j~\text{подходящих}~(A, p)
            \end{equation*}
            Лемма доказана.
       \end{proof}        
              
       \begin{lemma}
       \label{eq_some}
           Пусть $D=(V,E)$ --- граф, $G=(N,\Sigma, P, S)$ --- контекстно-свободная грамматика в нормальной форме Хомского. Тогда, для любых $i$, $j$, $k$ и нетерминала $A\in N$
       верно следующее: либо существует путь $i\pi j$ подходящий для пары $(A, k)$, такой что $a^k_{ij}(A) = w(\pi)$, либо не существует путей подходящих для пары $(A, k)$ и $a^k_{ij}(A) = \infty$.
       \end{lemma}
       \begin{proof}
            По индукции.\\
            \textbf{База:} по построению $a^1$ = $a$. Все деревья вывода высоты 1 соотвествуют каждое своему ребру.\\
            \textbf{Переход:} Пусть p > 1. Предположим, что утверждение леммы выполняется для $k < p$. Рассмотрим $a^p_{i,j}(A)$. Из \eqref{apply_plus} следует, что выполняется как минимум одно из следующих выражений:
            Рассмотрим случаи:\\
        
            \begin{equation*}
               a^p_{i,j}(A) = a^{p-1}_{i,j}(A) < \infty
            \end{equation*}
            По индукционнному предположению $a^{p-1}_{i,j}(A) < \infty$ тогда и только тогда, когда существует путь $i\pi j$ полходящий для пары $(A, p-1)$, он такжн будет подходящим для пары $(A, p)$.
            
            \begin{equation*}
               a^p_{i,j}(A) = (a^{p-1}\times a^{p-1})_{i,j}(A) < \infty
            \end{equation*}
            Из \eqref{apply_times} следует, что $(a^{p-1}\times a^{p-1})_{i,j}(A) < \infty$ тогда и только тогда, когда существуют такие $r$, $B$ и $C$, что $a^{p-1}_{ir}(B) < \infty$, $a^{p-1}_{rj}(C) < \infty$ и $A~\rightarrow~BC~\in~P$. Также, для некоторых таких $r$, $B$ и $C$ выполняется $a^{p}_{ij}(A) = a^{p-1}_{ir}(B) + a^{p-1}_{rj}(C)$. По индукционному предположению существуют пути $i\pi_1 r$ и $r\pi_2 j$ подходящие для $(B, p-1)$ и $(C, p-1)$ соответственно, а следовательно и $T_B$ --- дерево вывода $l(\pi_1)$ в $G_B$ и $T_C$ --- дерево вывода $l(\pi_2)$ в $G_C$ соответственно. Конкатенация $i\pi_1 r$ и $r\pi_2 j$ образует путь $i\pi j$, такой что $l(\pi)$ это конкатенция $l(\pi_1)$ и $l(\pi_2)$. При этом дерево, где в корне A, левое поддерево это $T_B$, а правое --- $T_C$, будет является деревом вывода $l(pi)$ в $G_A$б так как $A\rightarrow BC \in P$. Следовательно $i\pi j$ подходящий для пары $(A, k)$ и верно следующее:
            \begin{align*}
               a^{p}_{i,j}(A) &=\\
                              &= (a^{p-1}\times a^{p-1})_{i,j}(A)\\
                              &= a^{p-1}_{ir}(B) + a^{p-1}_{rj}(C)\\
                              &= w(\pi_1) + w(\pi_2)\\
                              &= w(\pi)
            \end{align*}
            В этом случае индукционный переход также выполняется.
            \begin{equation*}
               a^p_{i,j}(A) = a^{p-1}_{i,j}(A) = (a^{p-1}\times a^{p-1})_{i,j}(A) = \infty
            \end{equation*}
            По индукционному предположению не существует $i\pi j$ подходящего для $(A, p-1)$. Предположим, что существует путь $i\pi j$ подходящий для $(A, p)$ и неподходящий для $(A, p-1)$. Рассмотрим дерево вывода $l(\pi)$ в $G_A$ высоты $h=p$. Так как грамматика в нормальной форме Хомского, у корня есть ровно 2 ребенка. Они являются корнями подеревьев, являющихся по определению деревьями вывода некоторых строк $l_{left}$ в $G_B$ и $l_{right}$ в $G_C$ соответственно, высота каждого из этих деревьев строго меньше $p$, Так как конкатенация $l_{left}$ и $l_{right}$ это $l(\pi)$, $i\pi j$ можно разбить на $i\pi_{left} r$ и $r\pi_{right} j$ так, чтобы выполнялось $l(\pi_{left}) = l_{left}$ и $l(\pi_{right}) = l_{right}$. Таким образом, $i\pi_{left} r$ походящий для $(B, p-1)$, а $r\pi{_right} j$ подходящий для $(C, p-1)$. Следовательно, по индукционному предположению, $a^{p-1}_{ir}(B) < \infty$ и $a^{p-1}_{rj}(C) < \infty$. Так как $A\rightarrow BC \in P$, по определению $(\times)$, $a^p_{i,j}(A) \le a^{p-1}_{ir}(B) + a^{p-1}_{rj}(C) < \infty$. Это противоречит предположениюо условю $a^p_{i,j}(A) = \infty$, таким образом предположение о существовании пути $i\pi j$ подходящего для $(A, p)$ и неподходящего для $(A, p-1)$, неверно.
            
            Таким образом утверждение леммы выполняетнся при p, если оно выполняется при k < p. Лемма доказана.            
       \end{proof}
        
        \begin{theorem}
        \label{th_correctness}
            Пусть $D=(V,E)$ --- граф без циклов отрицательного веса, $G=(N,\Sigma, P, S)$ --- контекстно-свободная грамматика в нормальной форме Хомского. Тогда $a^{cf}_{ij}(A) = \infty$ если не существует пути $i\pi j$ такого, что существует дерево вывода строки $l(\pi)$ в $G_A$, и минимальной длине среди всех таких путей в противном случае.
        \end{theorem}
        \begin{proof}
            Предположим что не существует пути $i\pi j$ такого, что существует дерево вывода строки $l(\pi)$ в $G_A$. Тогда по~\cref{eq_some} для любого p $a^p_{ij}(A) = \infty$, следовательно $a^{cf}_{ij}(A) = \infty$.
            
            Предположим cуществует $i\pi j$ такой, что существует дерево вывода строки $l(\pi)$ в $G_A$. Так как в графе нет циклов отрицательного веса, найдется путь $i\pi_m j$ такой, что существует дерево вывода cтроки $l(\pi_m)$ в $G_A$ и при этом $w(\pi_m) \le w(\pi)$ для любого пути $i\pi j$ такого, что существует дерево вывода строки $l(\pi)$ в $G_A$. Пусть h --- высота некоторого дерева вывода строки $l(\pi_m)$ в $G_A$. По~\cref{le_any} $a^p_{ij}(A) \le w(\pi_m)$ для $p \ge h$, а по~\cref{eq_some} $a^p_{ij}(A) = w(\pi)$ для некоторого $i\pi j$ подходящего для $(A, p)$. По выбору $\pi_m$
         $w(\pi_m) \le w(\pi)$. Следовательно $w(\pi_m) \le a^p_{ij}(A) \le w(\pi_m)$, а значит $a^p_{ij}(A) = w(\pi_m)$ для $p \ge h$, следовательно $a^{cf}_{ij}(A) = w(\pi_m)$. Теорема доказана.
        \end{proof}
        
        \begin{theorem}
        \label{th_finiteness}
            Пусть $D=(V,E)$ --- граф без циклов отрицательного веса, $G=(N,\Sigma, P, S)$ --- контекстно-свободная грамматика в нормальной форме Хомского. Тогда $a^{cf}_{ij}(A) = a^p_{ij}(A)$ начиная с некоторого p.
        \end{theorem}
        \begin{proof}
            Для каждых $i$, $j$, $A$ построим множество путей $i\pi j$ таких что $w(\pi) = a^{cf}_{ij}(A)$ и существует дерево вывода $l(\pi)$ в $G_A$. Из каждого такого множества, кроме пустых выберем представителя и обозначим высоту некоторого соответствующего дерева вывода за $h^A_{ij} > 0$, а если множество пустое, то пусть $h^A_{ij} = 0$. Пусть $h_{max} = \max_{i, j, A}(h^A_{ij})$. Если $h_{max} = 0$, то в графе нет ни одного соответствующего пути и $a^1_{i,j}(A) = a^{cf}_{i,j}(A) = \infty$. Иначе, по~\cref{le_any}, $a^{h_{max}}_{ij}(A) \le a^{cf}_{ij}(A)$, а следовательно по определению $a^{cf}$ $a^{h_{max}}_{ij}(A) = a^{cf}_{ij}(A)$. Теорема доказана.     
        \end{proof}
    \subsection{Оценка времени работы}
    Оценка $O(|V|^5|N|^3)$. Умножение матриц в алгоритме выполняется за $O(|V|^3 * |N|^2)$. Оценка на количество итераций, $|V|^2|N|$, исходит из того что каждую итерацию в матрице будет фиксироваться одно значение. Рассуждения следует основывать на минимальных по высоте деревьях вывода кратчайших путей. !ДОПИСАТЬ ДОКАЗАТЕЛЬСТВО!
    
\section{Экспериментальное исследование}
    !РЕАЛИЗОВАТЬ АЛГОРИТМ И ЗАТЕСТИТЬ, ОПИСАТЬ ЭТО! 
 %У заключения нет номера главы
\section*{Заключение}
    В данной работе модифицирован алгоритм поиска пути с контекстно-свободными %ограничениями в помеченном графе\cite{DBLP:journals/corr/AzimovG17} для поиска %кратчайшего пути. Для достижения данной цели были решены следующие задачи:
    \begin{itemize}
        \item алгоритм модифицирован с помощью $(min, +)$-произведения;
        \begin{itemize}
            \item доказана корректность;
            \item дана оценка времени работы: $O(|V|^5|N|^3)$;
        \end{itemize}
        \item проведено экспериментальное исследование модифицированного алгоритма;
    \end{itemize}
    
\setmonofont[Mapping=tex-text]{CMU Typewriter Text}
\bibliographystyle{ugost2008ls}
\bibliography{diploma.bib}
\end{document}
